%! TeX program = lualatex
%! TeX root = main.tex

\def\basedir{/home/theammir/labs/asd/8/res}
\input{/home/theammir/labs/asd/template/template.tex}
\babelhyphenation{одно-зв'-яз-ний дво-зв'яз-ний лі-ній-ний}

\begin{document}
\thetitlepage{8}{ІМ-42}{Туров Андрій Володимирович}{28}

\taskdesc%
\begin{enumerate}
  \item Створити список з $n (n > 0)$ елементiв ($n$ вводиться з клавiатури), якщо iнша кiлькiсть елементiв не вказана у конкретному завданнi за варiантом.
  \item Тип ключiв (iнформацiйних полiв) задано за варiантом.
  \item Вид списку (черга, стек, дек, прямий однозв’язний лiнiйний список, обернений однозв’язний лiнiйний список, 
    двозв’язний лiнiйний список, одно\-зв’язний кiльцевий список, двозв’язний кiльцевий список)
  вибрати самостiйно з метою найбiльш доцiльного розв’язку поставленої за варiантом задачi.
  \item Створити функцiї (або процедури) для роботи зi списком (для створення, обробки, додавання чи видалення елементiв, виводу даних зi списку в консоль, звiльнення пам’ятi тощо).
  \item Значення елементiв списку взяти самостiйно такими, щоб можна було продемонструвати коректнiсть роботи алгоритму програми.
    Введення значень елементiв списку можна виконати довiльним способом (випадковi числа, формування значень за формулою, введення з файлу чи з клавiатури).
  \item Виконати над створеним списком дiї, вказанi за варiантом, та коректне звiльнення пам’ятi списку.
  \item \emph{При виконаннi заданих дiй, виводi значень елементiв та звiльненнi пам’ятi списку вважати, що довжина списку (кiлькiсть елементiв)
    невiдома на момент виконання цих дiй.} Тобто, не дозволяється зберiгати довжину списку як константу, змiнну чи додаткове поле.
\end{enumerate}

\taskspec%
Ключами елементiв списку є цiлi ненульовi числа. Кiлькiсть елементiв списку $n$ повинна бути кратною 10-ти,
а елементи у початковому списку розташовуватись iз чергуванням знакiв.
Перекомпонувати список, змiнюючи порядок чисел всерединi кожного десятка елементiв так, щоб спочатку йшли вiд'ємнi числа
цього десятка елементiв, а за ними --- додатнi, не використовуючи додаткових структур даних, крiм простих змiнних
(тобто <<на тому ж мiсцi>>).

\codetext{c}

\tasktest%
\testcase{./main -1 2 -3 4 -5 6 -7 8 -9 10}
\testcase{./main 1 -20 3 -41 5 -62 7 -83 9 -104}
\testcase{./main}

\conclusion%
Використав однозв'язний список для того щоб динамічно зберегти довільну кількість елементів.
Під час виконання програми групував вузли по 10 штук, змінюючи зв'язки між ними всередині цих груп, щоб досягти нового порядку.

\end{document}

% vim: ts=2: sw=2
